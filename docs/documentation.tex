\documentclass[12pt]{article}

\usepackage[T1]{fontenc}
\usepackage{geometry}
\usepackage{graphicx}
\usepackage{tocbibind}
\usepackage{parskip}
\usepackage{titlesec}

\titleformat{\subsection}[runin]{\normalfont\bfseries}{\thesubsection}{1em}{}[]
\titlespacing*{\subsection}{0pt}{\baselineskip}{\baselineskip}

\setlength{\parskip}{\baselineskip}
\setlength{\parindent}{2em}
\setcounter{tocdepth}{3}

\makeatletter
\renewcommand\tableofcontents{%
	\@starttoc{toc}%
}
\makeatother

\title{Kedarium}
\author{Robin Patrik Sloup}

\begin{document}
\begin{titlepage}
	\begin{center}
		{\Large Gymnázium a Střední Odborná Škola\par}
		\vspace{0.2cm}
		ul. Mládežníků 1115, Rokycany
	\end{center}
	\begin{center}
		\vspace{1cm}
		\includegraphics[height=3cm]{school-logo.png}\\
	\end{center}
	\vspace*{\fill}
	\begin{center}
		{\Huge \textbf{Kedarium,\\Grafický Engine}\par}
		\vspace{0.3cm}
		{\Large Maturitní Práce\par}
		\vspace{1.5cm}
	\end{center}
	\vspace*{\fill}
	\begin{center}
		\begin{minipage}[t]{0.45\textwidth}
			\centering
			\textbf{Autor Práce:}\\
			Robin Patrik Sloup
		\end{minipage}%
		\begin{minipage}[t]{0.45\textwidth}
			\centering
			\textbf{Vedoucí Práce:}\\
			Mgr. Jitka Fürbacherová
		\end{minipage}
	\end{center}
	\vspace{2cm}
	\begin{center}
		Rokycany 2024
	\end{center}
	\thispagestyle{empty}
\end{titlepage}
\setcounter{page}{2}
\section{Prohlášení}
Prohlašuji, že jsem maturitní práci s názvem „Kedarium, Grafický Engine“ vypracoval(a) samostatně pod vedením Jitky Fürbacherové a uvedl(a) v seznamu literatury všechny použité literární a odborné
zdroje. Dále prohlašuji, že citace použitých pramenů je úplná a že jsem v práci neporušil(a)
autorská práva (ve smyslu zákona č. 121/2000 Sb. O právu autorském, o právech souvisejících
s právem autorským).

\vspace{24pt}
\noindent
V Rokycanech dne 11. 9. 2001
\pagebreak

\section{Poděkování}
\pagebreak

\section{Obsah}
\tableofcontents
\pagebreak

\section{Abstrakt}

\section{Úvod}

\section{Analýza}

V této kapitole se zaměřím na to, jaká grafická rozhraní jsem mohl zvolit, zhodnocení jejich výhod a nevýhod a následně vysvětluje, proč jsem se rozhodl pro konkrétní řešení. Dalším krokem je aplikace stejného postupu při vysvětlení výběru programovacího jazyka, který jsem zvolil pro tento projekt. Po vysvětlení volby nástrojů, rozdělím grafické koncepty, které projekt využívá, do jednotlivých celků, které se budou vyskytovat v celé práci. Všechny tyto prvky stručně shrnu a následně rozvedu, jak fungují v reálné aplikaci.

Ke konci mé analýzy se zaměřím na popis matematických konceptů, jež jsou v tomto projektu využity. Tyto koncepty se využívají primárně pro práci s prostorem. Nalézají ale i jiná využití, ke kterým se dostanu v průběhu analýzy.

\subsection{Volba grafického rozhraní}

\subsection{OpenGL}

\subsection{Volba programovacího jazyka}

\subsection{Grafické koncepty}

\subsubsection{Kamera}

\subsubsection{Okno}

\subsubsection{Vertex Array Object}

\subsubsection{Vertex Buffer Object}

\subsubsection{Element Buffer Object}

\subsubsection{Textura}

\subsubsection{Světlo}

\subsubsection{GUI}

\subsection{Matematické koncepty}

\subsubsection{Vektor}

\subsubsection{Matice}

\section{Implementace}

\section{Testování}

\section{Uživatelská příručka}

\subsection{Windows}

\subsection{MacOS}

\subsection{Linux}

\section{Závěr}

\section{Seznam použité literatury}

\end{document}
